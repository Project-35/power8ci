\documentclass[10pt,onecolumn,journal,draftclsnofoot]{IEEEtran}
\usepackage[margin=0.75in]{geometry}
\usepackage{listings}
\usepackage{color}
\usepackage{longtable}
\usepackage{graphicx}
\usepackage{float}
\usepackage{tabu}
\usepackage{enumitem}
\usepackage{courier}
\usepackage{hyperref}
\usepackage{parskip}
\definecolor{dkgreen}{rgb}{0,0.6,0}
\definecolor{gray}{rgb}{0.5,0.5,0.5}
\definecolor{mauve}{rgb}{0.58,0,0.82}

\lstset{frame=none,
language=C,
columns=flexible,
numberstyle=\tiny\color{gray},
keywordstyle=\color{blue},
commentstyle=\color{dkgreen},
stringstyle=\color{mauve},
breaklines=true,
breakatwhitespace=true,
tabsize=4,
showstringspaces=false,
basicstyle=\ttfamily
}

\setlength{\parindent}{0cm}

\begin{document}
\begin{titlepage}
  \pagenumbering{gobble}
  \title{POWER8 Continuous Integration\\ Spring Midterm Progress Report}
  \author{Leon Leighton, Thomas Olson, Derek Wong\\Project 35}
  \date{May 15, 2017}
  \maketitle
  \vspace{4cm}
  \begin{abstract}
  \noindent This document contains an overview of the progress of the POWER8 Continuous Integration project.
    It includes the goals and purpose of the project, the current status, items remaining to be done, 
    and a discussion of problems we have encountered. 
 \end{abstract}
\end{titlepage}

\pagenumbering{arabic}
\tableofcontents
\clearpage

\section{Project Goals}
The primary goal of this project is to create a continuous integration (CI) system for IBM's POWER8 architecture.
This CI system will be available to open source software projects, giving them the ability to build and test their software on POWER8 without the need for them to acquire new hardware.
We will be utilizing the OSU Open Source Lab's POWER8 OpenStack cluster to deploy our project and run the builds and tests.
We also aim to make the system easy to use while still giving users the ability to customize their build environment.
Our focus will be on interacting with GitHub to trigger new builds as developers commit new changes to their code.
Users will be able to access the status of their builds to gain information about build or test failures.
They will also be able to download the binaries built by the system.
Our project will also be open source and hosted on GitHub which will allow others to replicate our project on their own systems.

\section{Current Progress}
Our project is currently feature complete with a few bugs and requested changes remaining to be resolved.
We currently have two virtual machines running on the OSL's POWER8 OpenStack cluster.
One serves as the Jenkins and web host, and the other serves as a Docker host. 
In addition, at any given time, multiple build VMs may be running to service Jenkins jobs that have requested to be built in a separate virtual machine rather than in a Docker container.
We use Ansible to install and configure Jenkins, Jenkins plugins, and our web site. 
Jenkins is the primary means of automating builds.
Depending on options selected by the user, Jenkins will direct the Docker host to start a container or direct OpenStack to start
a new virtual machine. 
Currently, Docker is the default if no option is selected by the user.

We make extensive use of existing Jenkins plugins.
The GitHub OAuth Plugin is used for authentication allowing our users to login to our service with their GitHub account.
For authorization, we are using the Matrix Authorization Strategy Plugin to give users the permissions that they need to run their builds and tests.
The Build Monitor, Embeddable Build Status, and the Email Notification plugins are used to communicate the status of Jenkins jobs to the user.
The Embeddable Build Status Plugin is used to create a GitHub Badge and will allow developers to see their build status on their project's GitHub page.
The Email Notification Plugin can be used to send an email to the developers whenever a build is broken.
The Build Monitor Plugin can be used to create a view that can give a user an overview of all their projects and their statuses.

To provide isolation between Jenkins jobs we use the Docker, CloudBees Docker Build and Publish, OpenStack, and Job Restriction plugins. 
The Docker and CloudBees Docker Build and Publish plugins enable the use of either a pre-built container image, or a user supplied Dockerfile.
The OpenStack Plugin allows a user to run their build in a completely separate VM which provides the maximum amount of isolation between project builds. 
The Job Restriction Plugin allows us to enforce building in either a Docker container or a VM, and allows us to prevent building jobs on the Jenkins instance itself. 


\section{Remaining Items}
We have a few items remaining to be done before the end of the capstone project.
We expect to have all of the following items done soon, but they are not complete at the time of writing this progress report.
The first item is to use the Apache web server as a reverse proxy to the Jenkins interface.
This change would allow us to use TLS to secure the login to Jenkins, as well as remove the need to use an alternate port number. 
That is, instead of using http://power-ci.osuosl.org:8080 to access Jenkins, one would use https://power-ci.osuosl.org/jenkins. 
Another item is to use our web site, https://power-ci.osuosl.org, to host basic documentation about the use of the system.
This would include information about how to request access and what abilities our service provides.
Also, relevant links to existing Jenkins documentation will be provided. 
A third item to be completed is the onboarding process itself. 
After discussions with IBM and the OSL, we will be making a change to the existing PowerLinux / OpenPOWER Request Form at http://osuosl.org/services/powerdev/request\_hosting/ to allow projects the ability to request access to the POWER8 CI system and get the GitHub username that should be added to the system. 
This request will go to the OSL and IBM for approval before access is granted.
The final item that we are in the process of, but remains to be completed, is to continue to build more complicated jobs to find and resolve any bugs.
For obvious reasons, we started with very simple Jenkins jobs to ensure the basic functioning of our CI system.
However, more complicated builds enable us refine needed components of our Docker container and virtual machine images, as well as any permission issues within Jenkins itself.


\section{Problems and Solutions}

\section{Retrospective}
\begin{center}
	\begin{tabular}{| p{0.3\linewidth} | p{0.3\linewidth} | p{0.3\linewidth} |}\hline
		Positives & Deltas & Actions \\ \hline
		Communication with client & Clarify design elements 	& Revise design document \\ \hline
		Communication with OSL & Lack of feedback between team members while writing documents & Share and request feedback earlier in writing process \\ \hline
		Communication between team members	& Some assignments rushed & Begin work earlier and keep each other up to date. \\ \hline
		Completed and turned in required documents & Did not make full use of clients availability & Ask for help or feedback from client earlier in process \\ \hline
	\end{tabular}
\end{center}

\end{document}
