\documentclass[10pt,onecolumn,journal,draftclsnofoot]{IEEEtran}
\usepackage[margin=0.75in]{geometry}
\usepackage{listings}
\usepackage{color}
\usepackage{longtable}
\usepackage{graphicx}
\usepackage{float}
\usepackage{tabu}
\usepackage{enumitem}
\usepackage{courier}
\usepackage{hyperref}
\usepackage{parskip}
\definecolor{dkgreen}{rgb}{0,0.6,0}
\definecolor{gray}{rgb}{0.5,0.5,0.5}
\definecolor{mauve}{rgb}{0.58,0,0.82}

\lstset{frame=none,
language=C,
columns=flexible,
numberstyle=\tiny\color{gray},
keywordstyle=\color{blue},
commentstyle=\color{dkgreen},
stringstyle=\color{mauve},
breaklines=true,
breakatwhitespace=true,
tabsize=4,
showstringspaces=false,
basicstyle=\ttfamily
}

\setlength{\parindent}{0cm}

\begin{document}
\begin{titlepage}
  \pagenumbering{gobble}
  \title{POWER8 Continuous Integration\\ Spring Midterm Progress Report}
  \author{Leon Leighton, Thomas Olson, Derek Wong\\Project 35}
  \date{May 15, 2017}
  \maketitle
  \vspace{4cm}
  \begin{abstract}
  \noindent This document contains an overview of the progress of the POWER8 Continuous Integration project.
    It includes the goals and purpose of the project, the current status, items remaining to be done, 
    and a discussion of problems we have encountered. 
 \end{abstract}
\end{titlepage}

\pagenumbering{arabic}
\tableofcontents
\clearpage

\section{Project Goals}
The primary goal of this project is to create a continuous integration (CI) system for IBM's POWER8 architecture.
This CI system will be available to open source software projects, giving them the ability to build and test their software on POWER8 without the need for them to acquire new hardware.
We will be utilizing the OSU Open Source Lab's POWER8 OpenStack cluster to deploy our project and run the builds and tests.
We also aim to make the system easy to use while still giving users the ability to customize their build environment.
Our focus will be on interacting with GitHub to trigger new builds as developers commit new changes to their code.
Users will be able to access the status of their builds to gain information about build or test failures.
They will also be able to download the binaries built by the system.
Our project will also be open source and hosted on GitHub which will allow others to replicate our project on their own systems.

\section{Current Progress}
Our project is currently feature complete with a few bugs and requested changes remaining to be resolved.
We currently have two virtual machines running on the OSL's POWER8 OpenStack cluster. 
One serves as the Jenkins and web host, and the other serves as a Docker host. 
Jenkins is the primary means of automating builds.
Depending on options selected by the user, Jenkins will direct the Docker host to start a container or direct OpenStack to start
a new virtual machine. 
Currently, Docker is the default if no option is selected by the user.

We make extensive use of existing Jenkins plugins.
The GitHub OAuth Plugin is used for authentication.
This means that our users are able to login to our service with their GitHub account.
For authorization, we are using the Matrix Authorization Strategy Plugin to give users the permissions that they need to run their builds and tests.
The Build Monitor, Embeddable Build Status, and the Email Notification plugins are used to communicate the status of Jenkins jobs to the user.
The Embeddable Build Status is used to create a GitHub Badge and will allow developers to see their build status when they go into their GitHub page.
The Email Notification Plugin can be used to send an email to the developers whenever a build is broken.
The Build Monitor plugin can be used to create a view that can give a user an overview of all their projects and their statuses.


One of our goals is to have developers build their project in Docker Containers or Virtual Machine, so the Job Restriction Plugin is what we used to bind them to those options.
If no option is chose between the two, the project will build in a Docker Container by default.
The last plugin on my list is the Docker Plugin and that is installed so developers can build in Docker Containers.
Most of the work is done through Ansible.
The installation of our POWER8 CI is deployed from it.
We have setup Ansible to configure and install plugins into Jenkins.


\section{Remaining Items}

\section{Problems and Solutions}

\section{Retrospective}
\begin{center}
	\begin{tabular}{| p{0.3\linewidth} | p{0.3\linewidth} | p{0.3\linewidth} |}\hline
		Positives & Deltas & Actions \\ \hline
		Communication with client & Clarify design elements 	& Revise design document \\ \hline
		Communication with OSL & Lack of feedback between team members while writing documents & Share and request feedback earlier in writing process \\ \hline
		Communication between team members	& Some assignments rushed & Begin work earlier and keep each other up to date. \\ \hline
		Completed and turned in required documents & Did not make full use of clients availability & Ask for help or feedback from client earlier in process \\ \hline
	\end{tabular}
\end{center}

\end{document}
