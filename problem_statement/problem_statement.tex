\documentclass[10pt,letterpaper,onecolumn,draftclsnofoot]{IEEEtran}
\usepackage[margin=0.75in]{geometry}
\usepackage{listings}
\usepackage{color}
\usepackage{longtable}
\definecolor{dkgreen}{rgb}{0,0.6,0}
\definecolor{gray}{rgb}{0.5,0.5,0.5}
\definecolor{mauve}{rgb}{0.58,0,0.82}

\lstset{frame=tb,
  language=C,
  columns=flexible,
  numberstyle=\tiny\color{gray},
  keywordstyle=\color{blue},
  commentstyle=\color{dkgreen},
  stringstyle=\color{mauve},
  breaklines=true,
  breakatwhitespace=true,
  tabsize=4
}

\begin{document}
\begin{titlepage}
  \title{CS 461 --- Fall 2016 --- Problem Statement}
  \author{Leon Leighton, Thomas Olson, Derek Wong\\POWER8 Build and Test Infrastructure}
  \date{October 14, 2016}
  \maketitle
  \vspace{4cm}
  \begin{abstract}
  \noindent Open Source software projects have a need for a POWER8 build and test infrastructure.
  This project will provide this ability using Cloud and Continuous Integration technologies. 
  Users of this project will have an easy to use method to ensure that their software works correctly on POWER8 systems running the Linux operating system.
  \end{abstract}
\end{titlepage}

\section*{Problem Definition}
The introduction of IBM's POWER8 architecture broadens the range of possible solutions available
In the case of solutions designed to be architecture independent, a POWER8 Continous Integration system reduces the effort needed to port and test software.  
In both cases, a POWER8 CI system allows for change by change validation of the software's ability to build and function correctly on POWER8 systems. 
IBM's POWER8 architecture is designed for high performance and scalability which make it ideal for today's Cloud and Big Data workloads.
Much of the software operating in these environments is Open Source. 
While much of the work needed to enable software to compile on POWER8 has already been done at the compiler level, Open Source software developers still need a way to build and test their software on POWER8. 
Public Cloud-based solutions already exist to enable Open Source software developers to build and test their software on Intel's x86 architecture. 
This project will focus on providing an easy to use, public, Cloud-based, POWER8 build and test infrastructure for Open Source projects. 


\section*{Proposed Solution}
The Oregon State University Open Source Lab (OSL) currently has a POWER8 cluster running OpenStack.
We will use this POWER8 Cloud along with the Jenkins automation software to create a POWER8 build and test system for Open Source projects. 
This system will be able to use both Linux virtual machines and Docker containers to build the software and run associated tests. 
We will also focus on ease of use through simple configuration and by allowing developers to use Github, and possibly other, credentials for authentication.

The project itself will also be Open Source which will allow others to replicate our solution on Clouds other than the one we will be utilizing at the OSL\@. 
This will be accomplished by hosting the project on Github. 

During the Engineering Expo in the Spring we will be able to show various Open Source projects being built and tested on POWER8 using this infrastructure.


\section*{Performance Metrics}
We will measure the success of our project by having a functioning POWER8 build and test infrastructure that is easy to use by Open Source projects. 

\clearpage
\section*{Approval}

\noindent Gerrit Huizenga\hspace{0.7cm} \makebox[1.5in]{\hrulefill}\\\\\\
Leon Leighton\hspace{0.7cm} \makebox[1.5in]{\hrulefill}\\\\\\
Thomas Olson\hspace{0.3cm} \makebox[1.5in]{\hrulefill}\\\\\\
Derek Wong\hspace{0.7cm} \makebox[1.5in]{\hrulefill}\\\\\\


\end{document}
